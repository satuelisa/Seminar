\documentclass{article}	% Always compile at least twice for placements!
\usepackage[top=2cm, bottom=3cm, outer=1cm, inner=0cm]{geometry}
% -------------------
% Packages
% -------------------
\usepackage{
	anyfontsize,
	amssymb,
	graphicx,
	mathptmx,
	t1enc,
	url,
	changepage 
}
\graphicspath{{imgs/}}
\usepackage[dvipsnames]{xcolor}
\usepackage[none]{hyphenat}
\urlstyle{same}
\usepackage[pages=some]{background}
\usepackage[percent]{overpic}
\usepackage{fancyhdr}

% When using anyfont size, \fontsize{#1}{#2}, #1 should be 
% approximately 95% the size of #2.

% -------------------
% Colors
% -------------------

% -------------------
% Background Image
% -------------------
\backgroundsetup{
	scale=1,
	color=black,
	opacity=0.40,
	angle=0,
	contents={%
  		\begin{overpic}[scale=1.1]{poster_bg.png} % Background Image
	     	%\put(60,3){\includegraphics[scale=0.2]{nsf_logo.eps}}  % If NSF funded
  		\end{overpic}
	  }%
}

% -------------------
% Information
% -------------------

% University Name
\newcommand{\universityname}{Universidad Autónoma de Nuevo León}
% Year
\newcommand{\conferenceyear}{2}
\newcommand{\super}{do}
% Date
\newcommand{\conferencedate}{24 de septiembre a las 12:00 horas}
\newcommand{\dinnerdate}{WEEKDAY MONTH DAY}
% First Speaker
\newcommand{\openspeakername}{
Dra. Laura Cruz Reyes 

}
\newcommand{\openspeakeruniversity}{Instituto Tecnológico de Ciudad Madero}
% Second Speaker
\newcommand{\titulo}{Modelos matemáticos de soporte a la toma de decisiones para la reducción del impacto socio-económico de la epidemia de COVID-19}
% Website URL
\newcommand{\conferencepage}{\url{https://youtu.be/OtZ_uaJgio0}}

\renewcommand{\headrulewidth}{0pt}

\lhead{\hspace{0.1cm}\includegraphics[width=.18\textwidth]{pisis_gris.png}}

\lfoot{\hspace{1cm}\includegraphics[width=.22\textwidth]{logotipo-uanl-prefooter.png}}
\cfoot{\hspace{1cm}\includegraphics[width=.12\textwidth]{logotipo-vision-2030.png}}
\rfoot{\includegraphics[width=.26\textwidth]{fime_footer_gris.png}\hspace{0.2cm}}
%-----------------------------------------------------
% Poster Content
%-----------------------------------------------------
\pagestyle{fancy}
\begin{document}
\sloppy
\pagenumbering{gobble}

% Side Text
\BgThispage
\rotatebox[origin=c]{90}{%
\begin{minipage}{\textheight} \hspace{1.5cm}
        \begin{minipage}{2cm}
        \fontsize{58}{60} \selectfont  
        \conferenceyear\textsuperscript{\super}
        \end{minipage}
\fontsize{32}{34} \selectfont  
\hspace{1.5cm}\color{ForestGreen} Ciclo de Conferencias 2021 \\[0.1cm]
\phantom{|} \hspace{1.2in} \color{NavyBlue} Posgrado en Ingeniería de Sistemas \color{black} \par
\rule{9.4in}{0.2cm}
\end{minipage}
}
% Do not remove commented line
% Top Text
\begin{minipage}[l][.7\textheight][t]{.90\linewidth} \centering \vspace{-1.5in}
\begin{adjustwidth}{50pt}{100pt}
\begin{center}
\fontsize{20}{45} \selectfont  
\textbf{\universityname} \par\vspace{-0.5cm}
\textbf{Facultad de Ingeniería Mecánica y Eléctrica} \par\vspace{0.3cm}
\fontsize{24}{26} \selectfont  

% Conferencista
\textbf{Ponente:} \par\vspace{0.2cm}
\fontsize{20}{20} \selectfont  
\textbf{\openspeakername} \par\vspace{0.32cm}
\fontsize{20}{20} \selectfont  
\textsf{\openspeakeruniversity} \par\vspace{0.3cm}

% Trabajo

\fontsize{18}{22} \selectfont  
\textbf{\titulo} \par\vspace{0.1cm}
\begin{center}
\fontsize{20}{26} \selectfont  
\textbf{Inicio: \conferencedate} \par\vspace{1cm}
\fontsize{14}{16} \selectfont
\includegraphics[scale=.22]{laura_cruz_reyes-modified.png}
%\includegraphics[scale=.07]{Patricia Liliana Cerda Pérez-modified.png}
%\includegraphics[scale=.3]{José Gregorio Alvarado Pérez-modified.png}
\includegraphics[scale=.5]{qr-code.png}
\par\vspace{0.5cm}
\textbf{\conferencepage} \par
\fontsize{16}{32} \selectfont 
\end{center}

% Resumen
\textbf{Resumen} \par\vspace{0.2cm}
\end{center}
\fontsize{15}{16} \selectfont 
\setlength{\parindent}{0pt}
\textsf{Se presenta un sistema inteligente de apoyo a la decisión desarrollado para realizar estudios epidemiológicos que soporten los esfuerzos por mitigar los efectos de la pandemia por COVID--19 en México. El sistema incluye un núcleo de modelos matemáticos y herramientas de software, conformado principalmente por modelos para la supervisión, control y estudios de propagación e impacto socioeconómico de la epidemia, así́ como el uso óptimo de los recursos del sistema de salud. El núcleo también incluye un modelo relacionado con la estimación de parámetros de los modelos epidemiológicos.
}

%\rule{0.72\textwidth}{0.1cm} \par\vspace{0.3cm}

\fontsize{10}{15} \selectfont  
\begin{center}
\fontsize{11}{11}
%\textbf{Una consideración importante.} \par
%\textbf{Cualquier otra consideración adicional.} \par

\end{center}
\end{adjustwidth}
\end{minipage}

\clearpage
\end{document}  