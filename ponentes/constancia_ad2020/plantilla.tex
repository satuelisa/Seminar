\documentclass{article}
\usepackage{blindtext}
\usepackage[utf8]{inputenc}
\usepackage{fancyhdr}
\usepackage{graphicx}
\usepackage{setspace}
\usepackage{transparent}
\usepackage{eso-pic}

\usepackage[letterpaper, landscape, left=3.5cm,right=1.5cm,top=1.7cm,bottom=0.5cm]{geometry}

\usepackage[pages=all]{background}
%configuración
\backgroundsetup{
 scale=1, %escala de la imagen, es recomendable que sea del mismo tamaño que el pdf
 color=black, %fondo a usar para transparencia
 opacity=100, %nivel de transparencia
 angle=0, %en caso de querer una rotación
 contents={%
  \includegraphics[width=\paperwidth,height=\paperheight]{wall-transparente.png} %nombre de la imagen a utilizar como fondo
 }%
}





\def\titulo{Título de la plática}
\def\ponente{Nombre del ponente}
\def\pisis{Posgrado en Ingeniería de Sistemas}
\def\seminario{Segundo Ciclo de Conferencias 2020}

% Coloca el mes con mayúscula inicial
\def\fecha{Fecha}

\def\organizador{Dra. Satu Elisa Schaeffer}
\def\coordinador{Dr. César E. Villarreal Rodríguez}

\def\uanl{\textsc{Universidad Autónoma de Nuevo León} }
\def\fime{\textsc{Facultad de Ingeniería Mecánica y Eléctrica} }

\doublespacing 

\begin{document}



\AddToShipoutPicture*{
	\put(10,240){
		\parbox[b][\paperheight]{\paperwidth}{%
			\vfill
			\centering
			\includegraphics[scale=.4]{pisis.png}%
			\vfill
		}
	}
	\put(-220,245){
		\parbox[b][\paperheight]{\paperwidth}{%
			\vfill
			\centering
			\includegraphics[scale=.4]{uanl-color.png}%
			\vfill
		}
	}
\put(250,250){
	\parbox[b][\paperheight]{\paperwidth}{%
		\vfill
		\centering
		\includegraphics[scale=.4]{fime-completo.png}%
		\vfill
	}
}
	\put(0,0){%
		\transparent{0.7}\textcolor{white}{\rule{\paperwidth}{\paperheight}}
	}
}

\vspace*{1.2 cm}
\begin{center}
\begin{Large}
La \fime y la \\ \uanl \\
otorgan la presente \\ \vspace*{0.2cm}
\end{Large}

\vspace*{0.5cm}
\begin{huge}
 \textsc{CONSTANCIA} 
\end{huge}
\vspace*{0.7cm}
 

 \begin{Large} al \end{Large} 
\begin{LARGE}
\textbf{\ponente} 
\end{LARGE}

\vspace*{0.5cm}
\begin{Large}
por su participación como ponente del trabajo \textbf{\titulo}, el día \textbf{\fecha} en el \textbf{\seminario} del \pisis. %\\~\\ 


\vspace*{3.7 cm}
\begin{tabular}{p{37mm}p{21mm}p{12mm}p{21mm}p{37mm}}	
	\cline{1-2} \cline{4-5}
	\multicolumn{2}{c}{\organizador} & & \multicolumn{2}{c}{\coordinador} \\
	\multicolumn{2}{c}{Organizadora del Seminario}   & & \multicolumn{2}{c}{Coordinador del Posgrado}   \\[17mm]
\end{tabular}
\end{Large}
\end{center}

\vspace*{-1.5cm}
\begin{flushleft}
\includegraphics[scale=0.2]{vision.png}
\end{flushleft}

\vspace*{-2cm}
\begin{flushright}
Ciudad Universitaria C.P 66455\\
San Nicolás de los Garza, Nuevo León, México.
\end{flushright}

\end{document}